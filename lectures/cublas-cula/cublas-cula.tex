%\documentclass{beamer}
\documentclass[handout]{beamer}
\usetheme{Marburg}
\useoutertheme{infolines}
\newcommand{\answers}{1}

\usepackage{amsmath}
\usepackage{caption}
\usepackage{color}
\usepackage{enumerate}
\usepackage{listings}
\usepackage{hyperref}
\usepackage{mathrsfs}
\usepackage{natbib}
\usepackage{url}

\providecommand{\all}{\ \forall \ }
\providecommand{\bs}{\backslash}
\providecommand{\e}{\varepsilon}
\providecommand{\E}{\ \exists \ }
\providecommand{\lm}[2]{\lim_{#1 \rightarrow #2}}
\providecommand{\m}[1]{\mathbb{#1}}
\providecommand{\nv}{{}^{-1}}
\providecommand{\ov}[1]{\overline{#1}}
\providecommand{\p}{\newpage}
\providecommand{\q}{$\quad$ \newline}
\providecommand{\rt}{\rightarrow}
\providecommand{\Rt}{\Rightarrow}
\providecommand{\vc}[1]{\boldsymbol{#1}}
\providecommand{\wh}[1]{\widehat{#1}}

\hypersetup{colorlinks,linkcolor=,urlcolor=blue}
\numberwithin{equation}{section}

\definecolor{dkgreen}{rgb}{0,0.6,0}
\definecolor{gray}{rgb}{0.5,0.5,0.5}
\definecolor{mauve}{rgb}{0.58,0,0.82}

\lstset{ 
  language=C,                % the language of the code
  basicstyle= \footnotesize,           % the size of the fonts that are used for the code
  numbers=left,
  numberfirstline=true,
  numbersep=5pt,                  % how far the line-numbers are from the code
  backgroundcolor=\color{white},      % choose the background color. You must add \usepackage{color}
  showspaces=false,               % show spaces adding particular underscores
  showstringspaces=false,         % underline spaces within strings
  showtabs=false,                 % show tabs within strings adding particular underscores
  frame=lrb,                   % adds a frame around the code
  rulecolor=\color{black},        % if not set, the frame-color may be changed on line-breaks within not-black text 
  tabsize=2,                      % sets default tabsize to 2 spaces
  captionpos=t,                   % sets the caption-position 
  breaklines=true,                % sets automatic line breaking
  breakatwhitespace=false,        % sets if automatic breaks should only happen at whitespace
  %title=\lstname,                   % show the filename of files included with \lstinputlisting;
  keywordstyle=\color{blue},          % keyword style
  commentstyle=\color{gray},       % comment style
  stringstyle=\color{dkgreen},         % string literal style
  escapeinside={\%*}{*)},            % if you want to add LaTeX within your code
  morekeywords={*, ...},               % if you want to add more keywords to the set
  xleftmargin=0.2in, % left horizontal offset of caption box
  xrightmargin=-.03in % right horizontal offset of caption box
}

%\DeclareCaptionFont{white}{\color{white}}
%\DeclareCaptionFormat{listing}{\parbox{\textwidth}{\colorbox{gray}{\parbox{\textwidth}{#1#2#3}}\vskip-0.05in}}
%\captionsetup[lstlisting]{format = listing, labelfont = white, textfont = white}
%For caption-free listings, comment out the 3 lines above and uncomment the 2 lines below.
 \captionsetup{labelformat = empty, labelsep = none}
 \lstset{frame = single}

\title{The CUBLAS and CULA libraries}
\author{Will Landau}
\date{October 28, 2013}
\institute{Iowa State University}

\begin{document}

\begin{frame}
\titlepage
 \end{frame}
 
 \begin{frame}
\frametitle{Outline}
\tableofcontents
\end{frame}
 
 \AtBeginSection[]
{
   \begin{frame}
       \frametitle{Outline}
       \tableofcontents[currentsection]
   \end{frame}
}

\section{CUBLAS overview}

\begin{frame}
\frametitle{CUBLAS}
\begin{itemize}
\item {\bf CUBLAS}: CUda Basic Linear Algebra Subroutines, the CUDA C implementation of BLAS.
\pause \item Consider scalars $\alpha, \beta$, vectors $\vc{x}$, $\vc{y}$, and matrices $A$, $B$, $C$.
\pause \item 3 ``levels of functionality":
\begin{itemize}
\pause \item Level 1: $\vc{y} \mapsto \alpha \vc{x} + y$ and other vector-vector routines.
\pause \item Level 2: $\vc{y} \mapsto \alpha A \vc{x} + \beta y$ and other vector-matrix routines.
\pause \item Level 3: $C \mapsto \alpha AB + \beta C$ and other matrix-matrix routines.
\end{itemize}
\end{itemize}
\end{frame}

\begin{frame}
\frametitle{Level 1 functions} \scriptsize
\begin{itemize}
\item Let $\alpha$ be a scalar, $\vc{x}$, $\vc{y}$, and $\vc{m}$ be vectors, $G = \begin{bmatrix} c & s \\ -s & c \end{bmatrix}$ be some $2 \times 2$ rotation matrix, and $H$ be an arbitrary $2 \times 2$ matrix,
\end{itemize}

\begin{center}
\pause \begin{tabular}{l|c|c}
in R & float & double \\ \hline
{\tt which.max($\vc{x}$)} & {\tt cublasIsamax()} &{\tt  cutlasIdamax()} \\
{\tt which.min($\vc{x}$)} &{\tt  cublasIsamin() }&{\tt  cublasIdamin()} \\
{\tt sum(abs($\vc{x}$))} &{\tt  cublasSasum() }& {\tt cublasDasum()} \\
{\tt $\alpha$ * $\vc{x}$ + $\vc{y}$-> $\vc{y}$} &{\tt  cublasSaxpy() }&{\tt cublasDaxpy()} \\
{\tt $\vc{x}$ -> $\vc{y}$} &{\tt  cublasScopy() }&{\tt  cublasDcopy()} \\
{\tt sum($\vc{x}$ * $\vc{y}$)} & {\tt cublasSdot()} & {\tt cublasDdot() }\\
{\tt sqrt(sum($\vc{x}^2$))} &{\tt  cublasSnrm2()} &{\tt  cublasDnrm2()} \\
{\tt $G$ \%*\% $\vc{x}$} & {\tt cublasSrot() }&{\tt  cublasDrot() }\\
{\tt $H$ \%*\% $\vc{x}$} &{\tt  cublasSrotm() }& {\tt cublasDrotm() }\\
{\tt $\alpha$ * $\vc{x}$ -> $\vc{x}$} & {\tt cublasSscal() }&{\tt  cublasDscal()} \\
{\tt $\vc{x}$ -> $\vc{m}$; $\vc{y}$ -> $\vc{x}$; $\vc{m}$ -> $\vc{y}$} & {\tt cublasSswap() }&{\tt  cublasDswap()}
\end{tabular}
\end{center}

\begin{itemize}
\pause \item Like everything in CUBLAS, there are also analogous functions for cuComplex and cuDoubleComplex types.
\end{itemize}
\end{frame}

\begin{frame}
\frametitle{Example level 2 functions}
\scriptsize
\begin{align*}
\alpha \text{op}(A) \cdot \vc{x} + \beta \vc{y} \mapsto \vc{y}
\end{align*}

\pause where

\begin{align*}
\text{op}(A) = \begin{cases}
A & \text{\tt transa == CUBLAS\_OP\_N} \\
A^T & \text{\tt transa == CUBLAS\_OP\_T} \\
A^H & \text{\tt transa == CUBLAS\_OP\_C} \\
\end{cases}
\end{align*}

\pause \begin{center}
\begin{tabular}{l|c|c}
type of matrix, $A$ & float & double \\ \hline
any $m \times n$ & {\tt cublasSgemv()} & {\tt cublasDgemv()} \\
banded $m \times n$ & {\tt cublasSgbmv()} & {\tt cublasDgbmv()} \\
symmetric, banded & {\tt cublasSbmv()} & {\tt cublasDbmv()} \\
symmetric, packed format & {\tt cublasSspmv()} & {\tt cublasDspmv()} \\
symmetric, triangular & {\tt cublasSsymv()} & {\tt cublasDsymv()}
\end{tabular}
\end{center}
\end{frame}

\begin{frame}
\frametitle{Example level 3 functions} \scriptsize
\begin{itemize}
\item {\tt cublasSgemm()} and {\tt cublasDgemm()}: for any compatible matrices $A$, $B$, and $C$,


\pause \begin{align*}
\alpha \cdot \text{op}(A) \text{op}(B) + \beta C \mapsto C
\end{align*}

\pause \item {\tt cublasSgemmBatched()} and {\tt cublasDgemmBatched()}: for arrays of compatible matrices $A[]$, $B[]$, and $C[]$,

\pause \begin{align*}
\alpha \cdot \text{op}(A[i]) \text{op}(B[i]) + \beta C[i] \mapsto C[i]
\end{align*}

\pause \item {\tt cublasStrsm()} and {\tt cublasDtrsm()} solve for $X$ when $A$ is triangular:

\pause \begin{align*}
\begin{cases}
\text{op}(A)X = \alpha B & \text{\tt trans == CUBLAS\_SIDE\_LEFT} \\
\text{Xop}(A) = \alpha B & \text{\tt trans == CUBLAS\_SIDE\_RIGHT} \\
\end{cases}
\end{align*}
\end{itemize}
\end{frame}


\section{Using CUBLAS}

\begin{frame}[fragile]
\frametitle{Implementation of matrices} \scriptsize
\begin{itemize}
\item Matrices stored in column major order in linear arrays of memory. 
\pause Array $A$,

\pause \begin{center}
\begin{tabular}{|c|c|c|c|c|c|c|c|c|c|c|c|} \hline
1 & 1 & 2 & 3 & 5 & 8 & 13 & 21 & 34 & 55 & 89 & 144 \\ \hline
\end{tabular} \q \q
\pause encodes matrix $B$,

\begin{align*}
\begin{bmatrix}
1 & 5 & 32 \\
1 & 8 & 55 \\
2 & 13 & 89 \\
3 & 21 & 144
\end{bmatrix}
\end{align*}

\end{center}

\pause \item Index by 
\begin{align*}
B[\text{row } i, \ \text{col } j] = A[j \cdot ld + i] 
\end{align*}

where $ld$ is the lead dimension of the matrix (column length for column major order matrices).


\pause \item Use a macro for indexing:
\begin{lstlisting}
#define IDX2F(i, j, ld) j * ld + i
\end{lstlisting}
\end{itemize}
\end{frame}

\begin{frame}[fragile]
\frametitle{CUBLAS context}
\begin{itemize}
\item For CUBLAS version $\ge $ 4.0, you must create a CUBLAS context:
\pause \begin{lstlisting}
cublasHandle_t handle;
cublasCreate(&handle);

// your code

cublasDestroy(handle);
\end{lstlisting}
\pause \item Pass {\tt handle} to every CUBLAS function in your code.
\pause \item This approach allows the user to use multiple host threads and multiple GPUs.
\end{itemize}
\end{frame}

\begin{frame}
\frametitle{CUBLAS helper functions}
\begin{itemize}
\item You don't actually need them, but you might see them:
\begin{itemize}
\item {\tt cublasSetVector()}
\item {\tt cublasGetVector()}
\item {\tt cublasSetMatrix()}
\item {\tt cublasGetMatrix()}
\end{itemize}

\end{itemize}

\end{frame}

\begin{frame}
\frametitle{Choosing the right header file}
\begin{itemize}
\item 2 choices of API 
\begin{itemize}
\pause \item {\tt cublas\_v2.h}: API for CUBLAS version 4.0 and above.
\pause \item {\tt cublas.h}: older API for programs written with CUBLAS version $<$ 4.0.
\end{itemize}
\pause \item Additions to newer API:
\begin{itemize}
\pause \item {\tt cublasCreate()} initializes the handle to the CUBLAS library context.
\pause \item Scalars can be passed by reference or by value to device functions.
\pause \item All CUBLAS functions return an error status, {\tt cublasStatus\_t}.
\pause \item {\tt cublasAlloc()} and {\tt cublasFree()} are deprecated. Use {\tt cudaMalloc()} and {\tt cudaFree()} instead.
\pause \item {\tt cublasSetKernelStream()} was renamed {\tt cublasSetStream()}.
\end{itemize}
\end{itemize}
\end{frame}

\begin{frame}[fragile]
\frametitle{Compiling with CUBLAS}
\begin{enumerate}
\item Include either {\tt cublas\_v2.h} or {\tt cublas.h} in your source
\pause \item Link the CUBLAS library with the {\tt -lcublas} flag.
\end{enumerate}

\begin{itemize}
\pause \item {\tt Example2.cu}:
\end{itemize}

\begin{lstlisting}
> nvcc -lcublas Example2.cu -o Example2.
> ./Example2
      1      7     13     19     25     31
      2      8     14     20     26     32
      3   1728    180    252    324    396
      4    160     16     22     28     34
      5    176     17     23     29     35
\end{lstlisting}
\end{frame}



\begin{frame}[fragile]
\lstset{basicstyle=\tiny}

\frametitle{{\tt Example2.cu}}
\begin{lstlisting}[name=ex2]
#include <stdio.h> 
#include <stdlib.h> 
#include <math.h> 
#include <cuda_runtime.h> 
#include <cublas_v2.h>
#define M 6
#define N 5
#define IDX2F(i,j,ld) (((j-1)*ld)+(i-1))

static __inline__ void modify (cublasHandle_t handle, float *m, int ldm, int n, int p,
  int q, float alpha, float beta){
  cublasSscal (handle, n - p+1, &alpha, &m[IDX2F(p,q,ldm)], ldm);
  cublasSscal (handle, ldm - p+1, &beta, &m[IDX2F(p,q,ldm)], 1);
}

int main (void){ 
  cudaError_t cudaStat;
  cublasStatus_t stat;
  cublasHandle_t handle;
  int i, j;
  float* devPtrA;
  float* a = 0;
  a = (float *)malloc (M * N * sizeof (*a)); 
  if (!a) {
    printf ("host memory allocation failed");
    return EXIT_FAILURE;
  }
\end{lstlisting}
\end{frame}


\begin{frame}[fragile]
\frametitle{{\tt Example2.cu}} \lstset{basicstyle=\tiny}

\begin{lstlisting}[name=ex2]
  for (j = 1; j <= N; j++) {
    for (i = 1; i <= M; i++) {
      a[IDX2F(i,j,M)] = (float)((i-1) * M + j); 
    }
  }

  cudaStat = cudaMalloc ((void**)&devPtrA, M*N*sizeof(*a)); 
  if ( cudaStat != cudaSuccess ) {
    printf ("device memory allocation failed");
    return EXIT_FAILURE;
  }
  
  stat = cublasCreate(&handle);
  if ( stat != CUBLAS_STATUS_SUCCESS ) {
    printf ("CUBLAS initialization failed\n");
    return EXIT_FAILURE; 
  }

  stat = cublasSetMatrix (M, N, sizeof(*a), a, M, devPtrA, M);
 
   if(stat != CUBLAS_STATUS_SUCCESS) {
    printf("data download failed");
    cudaFree(devPtrA);
    cublasDestroy(handle);
    return EXIT_FAILURE;
  }
 
\end{lstlisting}
\end{frame}


\begin{frame}[fragile]
\frametitle{{\tt Example2.cu}}  \lstset{basicstyle=\tiny}
\begin{lstlisting}[name=ex2]


  modify (handle, devPtrA, M, N, 2, 3, 16.0f, 12.0f);

  stat = cublasGetMatrix (M, N, sizeof(*a), devPtrA, M, a, M);
  if( stat != CUBLAS_STATUS_SUCCESS ) {
    printf ("data upload failed");
    cudaFree (devPtrA);
    cublasDestroy ( handle );
    return EXIT_FAILURE;
  }

  cudaFree ( devPtrA );
  cublasDestroy ( handle );

  for (j = 1; j <= N; j++) {
    for (i = 1; i <= M; i++) {
      printf ("%7.0f", a[IDX2F(i,j,M)]);
    }
    printf ( "\n" );
  }
  return EXIT_SUCCESS;
}
\end{lstlisting}
\end{frame}

\begin{frame}
\frametitle{Example: {\tt ols.cu}}

\begin{itemize}
\item I will attempt to solve the least squares problem,


\begin{align*}
\vc{y} = X \vc{ \beta} + \vc{\e} 
\end{align*}

by computing the solution,

\begin{align*}
\wh{\vc{\beta}} = (X^T X)^{-1}X^T \vc{y}
\end{align*}
\end{itemize}
\end{frame}


\begin{frame}[fragile]
\frametitle{Example: {\tt ols.cu}}  \lstset{basicstyle=\tiny}
\begin{lstlisting}[name=ols]
#include <stdio.h>
#include <stdlib.h>
#include <string.h>
#include <cuda_runtime.h>
#include <cublas_v2.h>
#include <cula.h>
#include <math.h>

#define I(i, j, ld) j * ld + i

#define CUDA_CALL(x) {if((x) != cudaSuccess){ \
  printf("CUDA error at %s:%d\n",__FILE__,__LINE__); \
  printf("  %s\n", cudaGetErrorString(cudaGetLastError())); \
  exit(EXIT_FAILURE);}} 

float rnorm(){
  float r1 = ((float) rand()) / ((float) RAND_MAX);
  float r2 = ((float) rand()) / ((float) RAND_MAX);
  return sqrt( -2 * log(r1) ) * cos(2 * 3.1415 * r2);
}

int main(){
  int i, j;
  int n = 10;
  int p = 3;
  int* ipiv;
  float k;
  float *X, *XtX, *XtY, *beta, *Y, *dX, *dXtX, *dXtY, *dbeta, *dY;
\end{lstlisting}
\end{frame}


\begin{frame}[fragile]
\frametitle{Example: {\tt ols.cu}}  \lstset{basicstyle=\tiny}
\begin{lstlisting}[name=ols]
  float *a, *b;
  a = (float*) malloc(sizeof(*X));
  b = (float*) malloc(sizeof(*X));
  *a = 1.0;
  *b = 0.0;

  cublasHandle_t handle;
  cublasCreate(&handle);

  X = (float*) malloc(n * p * sizeof(*X));
  XtX = (float*) malloc(p * p * sizeof(*X));
  XtY = (float*) malloc(p * sizeof(*X));
  beta = (float*) malloc(p * sizeof(*X));
  Y = (float*) malloc(n * sizeof(*X));
  
  CUDA_CALL(cudaMalloc((void**) &ipiv, p * p * sizeof(*ipiv)));
  CUDA_CALL(cudaMalloc((void**) &dX, n * p * sizeof(*X)));
  CUDA_CALL(cudaMalloc((void**) &dXtX, p * p * sizeof(*X)));
  CUDA_CALL(cudaMalloc((void**) &dXtY, p * sizeof(*X)));
  CUDA_CALL(cudaMalloc((void**) &dbeta, p * sizeof(*X)));
  CUDA_CALL(cudaMalloc((void**) &dY, n * sizeof(*X)));
 
\end{lstlisting}
\end{frame}



\begin{frame}[fragile]
\frametitle{Example: {\tt ols.cu}}  \lstset{basicstyle=\tiny}
\begin{lstlisting}[name=ols]
  printf("Y\t\tX\n");
  for(i = 0; i < n; i++){
    k = (float) i;
    X[I(i, 0, n)] = 1.0;
    X[I(i, 1, n)] = k / 10.0;
    X[I(i, 2, n)] = k * k / 10.0;  
    Y[i] = (k - 5.0) * (k - 2.3) / 3.0 + rnorm();
    
    printf("%0.2f\t\t", Y[i]);
    for(j = 0; j < p; j++){
      printf("%0.2f\t", X[I(i, j, n)]);
    } 
    printf("\n");
  }
  printf("\n");

  CUDA_CALL(cudaMemcpy(dX, X, n * p * sizeof(float), cudaMemcpyHostToDevice));
  CUDA_CALL(cudaMemcpy(dY, Y, n * sizeof(float), cudaMemcpyHostToDevice));

  cublasSgemm(handle, CUBLAS_OP_T, CUBLAS_OP_N, p, p, n, 
    a, dX, n, dX, n, b, dXtX, p);

  CUDA_CALL(cudaMemcpy(XtX, dXtX, p * p * sizeof(float), cudaMemcpyDeviceToHost));
\end{lstlisting}
\end{frame}

\begin{frame}[fragile]
\frametitle{Example: {\tt ols.cu}}  \lstset{basicstyle=\tiny}
\begin{lstlisting}[name=ols]
  printf("XtX\n");
  for(i = 0; i < p; i++){
    for(j = 0; j < p; j++){
      printf("%0.2f\t", XtX[I(i, j, p)]);
    }
    printf("\n");
  }
  printf("\n");
\end{lstlisting}
\end{frame}

\begin{frame}[fragile]
\frametitle{Output of code so far}  \lstset{basicstyle=\tiny}

\begin{lstlisting}[name=olsoutput]
> nvcc -I /usr/local/cula/include -L /usr/local/cula/lib64 -lcula_core -lcula_lapack -lcublas -lcudart ols.cu -o ols
> ./ols
Y		X
3.37		1.00	0.00	0.00	
1.94		1.00	0.10	0.10	
0.44		1.00	0.20	0.40	
-0.30		1.00	0.30	0.90	
-2.08		1.00	0.40	1.60	
-0.84		1.00	0.50	2.50	
-0.18		1.00	0.60	3.60	
3.40		1.00	0.70	4.90	
5.51		1.00	0.80	6.40	
7.39		1.00	0.90	8.10	

XtX
10.00	4.50	28.50	
4.50	2.85	20.25	
28.50	20.25	153.33	
\end{lstlisting}
\end{frame}


\begin{frame}[fragile]
\frametitle{Example: {\tt ols.cu}}
\begin{itemize}
\item We have $X^T X$, but which we need to invert in order to compute our solution,
\pause \begin{align*}
\wh{\vc{\beta}} = (X^T X)^{-1}X^T \vc{y}
\end{align*}
\pause \item But CUBLAS can only invert triangular matrices!
\end{itemize}
\end{frame}

\begin{frame}[fragile]
\frametitle{Enter CULA: CUDA LAPACK} \lstset{basicstyle=\tiny}
\begin{lstlisting}[name=ols]
  culaInitialize();
  
  culaDeviceSgetrf(p, p, dXtX, p, ipiv);
  culaDeviceSgetri(p, dXtX, p, ipiv);
  
  CUDA_CALL(cudaMemcpy(XtX, dXtX, p * p * sizeof(float), cudaMemcpyDeviceToHost));

  printf("XtX^(-1)\n");
  for(i = 0; i < p; i++){
    for(j = 0; j < p; j++){
      printf("%0.2f\t", XtX[I(i, j, p)]);
    }
    printf("\n");
  }
  printf("\n");

  cublasSgemm(handle, CUBLAS_OP_T, CUBLAS_OP_N, p, 1, n, 
    a, dX, n, dY, n, b, dXtY, p);

  cublasSgemv(handle, CUBLAS_OP_N, p, p, 
    a, dXtX, p, dXtY, 1, b, dbeta, 1);

  CUDA_CALL(cudaMemcpy(beta, dbeta, p * sizeof(float), cudaMemcpyDeviceToHost));

  printf("CUBLAS/CULA matrix algebra parameter estimates:\n");
  for(i = 0; i < p; i++){
    printf("beta_%i = %0.2f\n", i, beta[i]);
  }
  printf("\n");
\end{lstlisting}
\end{frame}

\begin{frame}[fragile]
\frametitle{CULA's {\tt culaSgels()} does least squares for you}  \lstset{basicstyle=\tiny}
\begin{lstlisting}[name=ols]
  culaSgels('N', n, p, 1, X, n, Y, n);

  printf("culaSgels Parameter estimates:\n");
  for(i = 0; i < p; i++){
    printf("beta_%i = %0.2f\n", i, Y[i]);
  }
  printf("\n");

  culaShutdown();
  cublasDestroy(handle);

  free(a);
  free(b);
  free(X);
  free(XtX);
  free(XtY);
  free(beta);
  free(Y);

  CUDA_CALL(cudaFree(dX));
  CUDA_CALL(cudaFree(dXtX));
  CUDA_CALL(cudaFree(dXtY));
  CUDA_CALL(cudaFree(dbeta));
  CUDA_CALL(cudaFree(dY));
}
\end{lstlisting}
\end{frame}

\begin{frame}[fragile]
\frametitle{Rest of the output}
\begin{lstlisting}[name=olsoutput]
XtX^(-1)
0.62	-2.59	0.23	
-2.59	16.55	-1.70	
0.23	-1.70	0.19	

CUBLAS/CULA matrix algebra parameter estimates:
beta_0 = 3.78
beta_1 = -25.53
beta_2 = 3.36

culaSgels Parameter estimates:
beta_0 = 3.78
beta_1 = -25.53
beta_2 = 3.36
\end{lstlisting}
\end{frame}

\section{CULA}

\begin{frame}
\frametitle{More on CULA}
\begin{itemize}
\item {\tt CULA}: the CUDA C implementation of LAPACK
\pause \item Features:
\begin{itemize}
\item More matrix algebra routines
\item Factorizations: LU, QR, RQ, QL, SVD, and Cholesky
\item Solving systems of linear equations (matrix inversion)
\item Least squares
\item Eigenvalue solvers
\end{itemize}
\pause \item Interfaces (collections of functions)
\begin{itemize}
\pause \item {\bf Standard}: users need not micromanage GPU memory or copy data to or from the GPU.
\pause \item {\bf Device}: users need explicitly to allocate GPU memory and copy to and from the GPU.
\end{itemize}
\pause \item Be careful of the standard interface functions: they're convenient, but they copy to and from the GPU with every call.
\end{itemize}
\end{frame}

\begin{frame}
\frametitle{CULA naming conventions}

\begin{align*}
\underbrace{\text{\Huge cula}}_\text{prefix} \underbrace{\text{\Huge S}}_\text{type} \underbrace{\text{\Huge ge}}_\text{matrix} \underbrace{\text{\Huge qrf}}_\text{routine}
\end{align*}

\begin{itemize} \scriptsize
\pause \item Prefix: {\tt cula} for standard interface, {\tt culaDevice} for device interface:
\pause \item Type: single precision ({\tt S}),  single precision complex ({\tt C}), double precision real ({\tt D}), or double precision complex ({\tt Z}).
\pause \item Matrix:
\begin{center}
\begin{tabular}{l|l}
{\tt bd} & Bidiagonal \\
{\tt ge} & General \\
{\tt gg} & General matrices, generalized problem \\
{\tt he} & Hermitian symmetric \\
{\tt or} & (Real) orthogonal \\
{\tt sb} & Symmetric, banded \\
{\tt sy} & Symmetric \\
{\tt tr} & Triangular \\
{\tt un} & (Complex) unitary
\end{tabular}
\end{center}
\end{itemize}
\end{frame}

\begin{frame}
\frametitle{CULA naming conventions}

\begin{align*}
\underbrace{\text{\Huge cula}}_\text{prefix} \underbrace{\text{\Huge S}}_\text{type} \underbrace{\text{\Huge ge}}_\text{matrix} \underbrace{\text{\Huge qrf}}_\text{routine}
\end{align*}
\small
\begin{itemize}
\item Routine:
\begin{center}
\begin{tabular}{l|l}
{\tt trf} & Triangular factorization \\
{\tt sv} & Factor a matrix and solve system of linear equations \\
{\tt qrf} & QR factorization without pivoting \\
{\tt svd} & Singular value decomposition \\
{\tt ls} & Solve over- or under-determined linear system
\end{tabular}
\end{center}
\item Consult the CULA manual for other routines.
\end{itemize}
\end{frame}

\begin{frame}[fragile]
\frametitle{Compiling with CULA}
\begin{itemize}
\item Include {\tt cula\_lapack.h} for the standard interface, {\tt cula\_lapack\_device.h} for the device interface, or {\tt cula.h} for both.
\pause \item Compile on impact1 with:
\begin{lstlisting}
nvcc -I /usr/local/cula/include -L /usr/local/cula/lib64 -lcula_core -lcula_lapack -lcublas -lcudart your_source.cu -o your_binary
\end{lstlisting}
\begin{itemize}
\pause \item {\tt -I /usr/local/cula/include} tells the compiler, {\tt nvcc}, where to find the header files.
\pause \item {\tt -L /usr/local/cula/lib64} tells {\tt nvcc} where the CULA library is (the 64-bit version in this case).
\pause \item {\tt -lcula\_core -lcula\_lapack -lcublas -lcudart} links the required libraries to your binary.
\end{itemize}
\end{itemize}
\end{frame}

\begin{frame}[fragile]
\frametitle{Minimal working example: {\tt mwe.cu}}
\begin{lstlisting}[name=mwe]
#include <cula.h>
#include <stdlib.h>
#include <stdio.h>

int main(){
  
  culaStatus s;
  s = culaInitialize();
  
  if(s != culaNoError)
    printf("%s\n", culaGetErrorInfo());
    
  /* ... Your code ... */
  
  culaShutdown();
}
\end{lstlisting}
\end{frame}

\begin{frame}[fragile]
\frametitle{Minimal working example: {\tt mwe.cu}}
\begin{lstlisting}[name=mwe]
> nvcc -I /usr/local/cula/include -L /usr/local/cula/lib64 -lcula_core -lcula_lapack -lcublas -lcudart mwe.cu -o mwe
> ./mwe
> 
\end{lstlisting}
\end{frame}



\begin{frame}
\frametitle{Outline}
\tableofcontents
\end{frame}


\begin{frame}
\frametitle{Resources} \small

\begin{itemize}
\item Guides:
\begin{enumerate}
 \item \href{http://docs.nvidia.com/cuda/pdf/CUDA_CUBLAS_Users_Guide.pdf}{�CUDA Toolkit 4.2 CUBLAS Library�}
\pause \item \href{http://www.culatools.com/cula_dense_programmers_guide/}{``CULA Programmer�s Guide". CULA Tools.}
\pause \item \href{http://geco.mines.edu/tesla/cula/CULAReferenceManual_1.1.pdf}{"CULA Reference Manual". CULA Tools.}
\end{enumerate}
\pause \item Code from today:
\begin{itemize}
\item \href{http://will-landau.com/gpu/Code/CUDA_C/CUBLAS/Example2/Example2.cu}{Example2.cu}
\item \href{http://will-landau.com/gpu/Code/CUDA_C/CULA/ols/ols.cu}{ols.cu}
\item \href{http://will-landau.com/gpu/Code/CUDA_C/CULA/mwe/mwe.cu}{mwe.cu}
\end{itemize}
\pause \item Other example code:
\begin{itemize}
\item \href{http://will-landau.com/gpu/Code/CUDA_C/CUBLAS/simpleCUBLAS/simpleCUBLAS.cpp}{simpleCUBLAS.cpp}
\item \href{http://will-landau.com/gpu/Code/CUDA_C/CULA/Argument_Errors/ae.cu}{ae.cu}
\item \href{http://will-landau.com/gpu/Code/CUDA_C/CULA/Data_Errors/de.cu}{de.cu}
\item \href{http://will-landau.com/gpu/Code/CUDA_C/CULA/deviceInterface/deviceInterface.c}{deviceInterface.c}
\item \href{http://will-landau.com/gpu/Code/CUDA_C/CULA/Library_Link_Check/ll.cu}{ll.cu}
\item \href{http://will-landau.com/gpu/Code/CUDA_C/CULA/systemSolve/systemSolve.c}{systemSolve.c}
\end{itemize}
\end{itemize}
\end{frame}


\begin{frame}
\frametitle{That's all for today.}
\begin{itemize}
\item Series materials are available at \url{http://will-landau.com/gpu}.
\end{itemize}
\end{frame}

\end{document}